\documentclass[french]{article}
\usepackage{ae, lmodern}
\usepackage[utf8]{inputenc}
\usepackage[T1]{fontenc}




\title{Simulation sur ordinateur \\ Première session : Rapport \\ Etude du caractère pseudo-aléatoire de $\pi$}
\author{Maazouz Mehdi, Caredda Giuliano \\ BA3 Info}
\date{28 Mai 2018}

\begin{document}
\maketitle
\newpage
\renewcommand{\contentsname}{Sommaire}
\tableofcontents
\newpage

\section{Introduction}
Le travail que nous devons présenter consiste à étudier le caractère pseudo-aléatoire des décimales de $\pi$
en utilisant les techniques vues au cours et ce, suivant une lois uniforme. \\
De plus, nous allons devoir nous servir de ces décimales pour implémenter un générateur de loi uniforme [0,1[ 
et de comparer ce dernier au générateur par défaut de Python. \\
Nous effectuerons 3 tests pour l'étude du caractère pseudo aléatoire et 3 autres tests pour la comparaison avec le générateur par défaut de Python. Ces derniers seront décris ci-dessous. \\
Nous nous servirons du langage Python pour effectuer nos différents tests.

\section{Etude de $\pi$ }
Nous allons désormais passer à l'étude du caractère pseudo-aléatoire des décimales de $\pi$
par l'intermédiaire de 3 tests différents.

\end{document}
